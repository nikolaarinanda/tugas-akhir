\newpage
\pagestyle{fancy}
\fancyhf{}
\fancyhead[R]{\thepage}
\chapter{PENDAHULUAN} \label{Bab I}

\section{Latar Belakang} \label{I.Latar Belakang}
% \textit{Mean Absolute Error} (MAE) \cite{Suryanto2019MAE}
% \lipsum[1-3] % Menampilkan paragraf 1 sampai 2 dari lorem ipsum

Perkembangan teknologi digital yang pesat telah membawa perubahan signifikan dalam cara manusia berinteraksi. Namun, di balik kemudahan yang dihadirkan seperti media sosial, permainan video, dan internet, muncul pula tantangan serius, terutama terkait dampaknya pada kesehatan mental. Teknologi digital dapat membawa dampak negatif yaang signifikan untuk kesehatan mental seperti kecanduan, gangguan tidur, depresi, dan kecemasa, terutama pada kelompok rentan seperti anak muda \cite{sembiring2024dampak}. Fenomena ini menjadi perhatian global seiring dengan meningkatnya jumlah pengguna digital dari berbagai usia.

Salah satu ancaman terbesar terhadap kesehatan mental di dunia maya adalah \textit{cyberbullying}. Ini merupakan tindakan perundungan yang terjadi di ranah digital, seringkali dalam bentuk komentar bernada negatif, hinaan, maupun pelecehan verbal \cite{putri2023cyberbullying}. Dampak \textit{cyberbullying} sangat serius, tidak hanya menyebabkan tekanan psikologis seperti stres dan gangguan tidur, tetapi juga dapat meningkatkan risiko masalah kesehatan mental yang lebih parah, termasuk depresi, kecemasan, dan, dalam kasus ekstrem, pikiran untuk bunuh diri pada korbannya \cite{kurniawan2024dampak}.

Kemajuan teknologi digital telah menghadirkan berbagai platform komunikasi yang memungkinkan interaksi antarindividu dari seluruh penjuru dunia. Platform-platform ini menjadi ruang utama bagi pengguna untuk berbagi informasi, mengekspresikan diri, serta menjalin relasi sosial secara luas \cite{sari2018komunikasi}. Salah satu platform yang mengalami pertumbuhan sangat pesat dalam beberapa tahun terakhir adalah TikTok. Aplikasi ini digunakan oleh jutaan orang dari berbagai usia dan latar belakang, menjadikannya fenomena global yang mengubah pola konsumsi serta produksi konten video pendek. Popularitas TikTok yang terus meningkat membuatnya menjadi salah satu aplikasi paling banyak diunduh di dunia pada tahun 2023 \cite{rosiana2023analisis}.

Maraknya komentar bernada negatif, hinaan, dan pelecehan verbal menjadi indikasi kuat adanya \textit{cyberbullying} di TikTok. Tindakan semacam ini tidak hanya mencemari ruang digital, tetapi juga dapat berdampak buruk pada kondisi psikologis korban. Korban dapat merasa malu, takut, dan terintimidasi oleh komentar yang bersifat menyerang \cite{aser2022fenomena}. Karakteristik komentar yang singkat, tidak baku, serta informal membuat proses deteksi \textit{cyberbullying} sangat sulit jika dilakukan secara manual, karena membutuhkan banyak sumber daya dan cenderung subjektif.

Mengatasi permasalahan ini, diperlukan solusi analisis sentimen yang mampu mengklasifikasikan komentar secara otomatis. Analisis sentimen adalah proses identifikasi opini atau emosi dalam teks, yang dapat dikategorikan menjadi sentimen positif, netral, atau negatif. Proses analisis yang optimal membutuhkan tahapan text processing yang sistematis, seperti pembersihan teks dari noise, tokenisasi, penghapusan stopword yang tidak relevan, serta konversi teks ke representasi numerik yang dapat diproses oleh sistem komputasi \cite{razi2017klasifikasi}.

Pendekatan modern dalam analisis sentimen banyak mengandalkan teknik Deep Learning. Metode ini, yang merupakan bagian dari pembelajaran mesin dengan jaringan saraf tiruan berlapis-lapis, memiliki kemampuan luar biasa dalam mempelajari pola kompleks dari data mentah, termasuk data teks, tanpa perlu fitur yang dirancang secara manual \cite{liao2017cnn}. Ini membuka jalan bagi sistem otomatis yang lebih canggih untuk memahami nuansa bahasa.

Di bawah payung deep learning, bidang natural language processing (NLP) memeng peranan krusial. NLP berfokus pada interaksi antara komputer dan bahasa manusia, memungkinkan mesin untuk memproses, menganalisis, memahami dan menghasilkan bahasa alami. Hal ini sangat esensial untuk mengurai makna di balik komentar-komentar pengguna dan mengidentifikasi sentimen yang terkandung di dalamnya \cite{razi2017klasifikasi}.

Beberapa penelitian terdahulu telah mencoba mengatasi masalah \textit{cyberbullying} dengan menggunakan arsitektur canggih seperti \textit{Bidirectional Encoder Representations from Transformers} (BERT) melalui proses fine-tuning pada dataset spesifik. Model-model berbasis transformer seperti BERT memang menunjukkan performa tinggi dalam banyak tugas \textit{Natural Text Processing} (NLP). Namun, hasil dari studi-studi tersebut seringkali menunjukkan kecenderungan overfitting, di mana model terlalu spesifik terhadap data pelatihan sehingga performanya buruk pada data baru yang belum pernah dilihat sebelumnya, terutama dengan ukuran dataset yang terbatas atau distribusi data yang tidak seimbang \cite{10468424}.

Melihat tantangan tersebut, arsitektur TextCNN muncul sebagai alternatif yang menjanjikan. Penelitian terdahulu terkait TextCNN telah membuktikan keunggulannya dalam mengekstraksi fitur penting dari rangkaian kata dan telah banyak digunakan dalam berbagai tugas klasifikasi teks, termasuk analisis sentimen dan deteksi spam atau misinformasi. Keunggulan utamanya adalah kemampuannya dalam mempelajari representasi fitur secara otomatis melalui filter konvolusi, yang efektif menangkap pola lokal dan global dalam teks \cite{kim2014convolutional}.

Berdasarkan uraian di atas, penelitian ini berfokus pada penerapan arsitektur TextCNN untuk membangun sistem klasifikasi sentimen komentar TikTok, khususnya yang mengandung indikasi \textit{cyberbullying}. Sistem ini akan menggunakan pendekatan binary classification untuk membedakan antara komentar \textit{cyberbullying} dan \textit{non-cyberbullying}, dengan harapan dapat mencapai generalisasi yang baik. Diharapkan hasil dari penelitian ini dapat memberikan kontribusi signifikan dalam mendukung sistem moderasi konten serta menciptakan lingkungan media sosial yang lebih sehat, aman, dan ramah bagi seluruh pengguna.

\section{Rumusan Masalah} \label{I.Rumusan Masalah}

Berdasarkan latar belakang yang telah diuraikan di atas, maka permasalahan penelitian dirumuskan sebagai berikut: \par

\begin{enumerate}[noitemsep]
    \item Bagaimana mengklasifikan komentar yang termasuk dan tidak termasuk \textit{cyberbullying} pada platform TikTok dengan arsitektur TextCNN
    \item bagaimana menguji performa model TextCNN dalam mengklasifikasikan komentar \textit{cyberbullying} dan yang tidak.
\end{enumerate}

\section{Tujuan Penelitian} \label{I.Tujuan}
Berdasarkan rumusan masalah yang telah diuraikan di atas, maka tujuan dari penelitian ini adalah: \par

\begin{enumerate}[noitemsep]
    \item Mengembangkan model klasifikasi untuk mendeteksi komentar yang termasuk dan tidak termasuk \textit{cyberbullying} pada platform TikTok menggunakan arsitektur TextCNN.
    \item Mengevaluasi performa model TextCNN dalam mengklasifikasikan komentar \textit{cyberbullying} dan komentar non-\textit{cyberbullying} berdasarkan metrik evaluasi seperti akurasi, presisi, recall, dan F1-score.
\end{enumerate}


\section{Batasan Masalah} \label{I.Batasan}
Adapun batasan masalah dari penelitian ini agar sesuai dengan yang diharapkan adalah sebagai berikut: \par

\begin{enumerate}[noitemsep]
    \item Penelitian ini hanya menggunakan data komentar yang berasal dari platform TikTok.
    \item Data komentar TikTok yang digunakan dalam penelitian ini diambil dari dataset yang telah dikumpulkan dan dipublikasikan oleh Bunga Aura Prameswari et al \cite{10468424}.
    \item Klasifikasi dilakukan secara biner, yaitu dengan fokus pada pendeteksian komentar yang mengandung indikasi \textit{cyberbullying} dan yang tidak.
    \item Data yang dianalisis berjumlah 1.508 komentar dan seluruhnya menggunakan bahasa Indonesia.
\end{enumerate}

\section{Manfaat Penelitian} \label{I.Manfaat}
Adapun manfaat yang diperoleh dari hasil penelitian ini adalah sebagai berikut: \par

\begin{enumerate}[noitemsep]
    \item Memberikan kontribusi dalam pengembangan sistem otomatis untuk mendeteksi komentar negatif dan \textit{cyberbullying} di media sosial, khususnya pada platform TikTok.
    \item Menjadi acuan atau referensi dalam penerapan arsitektur TextCNN untuk klasifikasi teks pendek, terutama pada komentar media sosial yang memiliki karakteristik bahasa informal dan ringkas.
    \item Membantu mengurangi penyebaran komentar berunsur \textit{cyberbullying} di platform digital melalui pengembangan sistem analisis teks berbasis pembelajaran mesin.
\end{enumerate}


\section{Sistematika Penulisan} \label{I.Sistematika}
Sistematika penulisan berisi pembahasan apa yang akan ditulis disetiap Bab. Sistematika pada umumnya berupa paragraf yang setiap paragraf mencerminkan bahasan setiap Bab. \par

\noindent\textbf{Bab I}

Bab ini membahas latar belakang yang melandasi penelitian, perumusan masalah yang ingin diselesaikan, serta tujuan yang ingin dicapai. Selain itu, dijelaskan juga batasan masalah, manfaat dari penelitian, dan sistematika penulisan sebagai panduan struktur laporan.

\noindent\textbf{Bab II}

Bab ini menguraikan teori-teori yang menjadi dasar penelitian, seperti konsep cyberbullying, analisis sentimen, dan text processing. Juga dijelaskan arsitektur model Text CNN yang digunakan dalam penelitian ini.

\noindent\textbf{Bab III}

Bab ini menjelaskan metode yang digunakan dalam penelitian, mulai dari teknik pengumpulan data hingga tahapan text processing. Selain itu, dibahas pula perancangan model Text CNN serta metode evaluasi performa model.

\noindent\textbf{Bab IV}

Bab ini menyajikan hasil pelatihan dan pengujian model yang telah dibangun. Dilengkapi dengan analisis performa dan interpretasi hasil klasifikasi sentimen untuk menilai keberhasilan model.

\noindent\textbf{Bab V}

Bab terakhir ini berisi kesimpulan dari hasil penelitian yang telah dilakukan. Penulis juga memberikan saran sebagai masukan untuk pengembangan penelitian di masa mendatang.